\documentclass[12pt,-letter paper]{article}

%\usepackage[left=1.5in,right=1in,top=1in,bottom=1in]{geometry}
%\usepackage[left=1.5in,right=1in]{geometry}
%\usepackage{geometry}
%\makeatletter%
%\textheight     243.5mm
%\textwidth      183.0mm
%\textwidth=31pc%
%\textheight=48pc
\usepackage{lipsum}% this package is included to get dummy paragraphs for sample purpose.
\usepackage{ulem}
\usepackage{alltt}
\usepackage{tfrupee}
\usepackage[anticlockwise,figuresright]{rotating}
\usepackage{pstricks}
\usepackage{wrapfig}
\usepackage{pstcol,pst-grad}
 \usepackage{bm}
\usepackage{enumitem}
\usepackage{listings}
    \usepackage{color}                                            %%
    \usepackage{array}                                            %%
    \usepackage{longtable}                                        %%
    \usepackage{calc}                                             %%
    \usepackage{multirow}                                         %%
    \usepackage{hhline}                                           %%
    \usepackage{ifthen}                                           %%
  %optionally (for landscape tables embedded in another document): %%
    \usepackage{lscape}     
    \usepackage{gensymb}     
    \usepackage{tabularx}
\usepackage{ifthen}%
\usepackage{amsmath}%
\usepackage{color}%
\usepackage{float}%
\usepackage{graphicx}%
%\usepackage[right]{showlabels}%
\usepackage{boites}%
\usepackage{boites_exemples}%
\usepackage{graphicx,pstricks}
%\usepackage{enumerate}%
\usepackage{latexsym}
\usepackage[fleqn]{mathtools}
\usepackage{amssymb,amsfonts,amsthm}
\usepackage{mathrsfs,makeidx,listings,verbatim,moreverb}
%%\usepackage{amsthm,mathrsfs,makeidx,listings,verbatim,moreverb}
%\let\eqref\ref%  updated on 20th April 2017

\usepackage{hyperref}%
%\usepackage[dvips]{hyperref}%
\hypersetup{bookmarksopen=false}%
\usepackage{breakurl}%
\usepackage{tkz-euclide} % loads  TikZ and tkz-base

\newcommand{\solution}{\noindent \textbf{Solution: }}
\providecommand{\mbf}{\mathbf}
\providecommand{\rank}{\text{rank}}
\providecommand{\pr}[1]{\ensuremath{\Pr\left(#1\right)}}
\providecommand{\qfunc}[1]{\ensuremath{Q\left(#1\right)}}
\providecommand{\sbrak}[1]{\ensuremath{{}\left[#1\right]}}
\providecommand{\lsbrak}[1]{\ensuremath{{}\left[#1\right.}}
\providecommand{\rsbrak}[1]{\ensuremath{{}\left.#1\right]}}
\providecommand{\brak}[1]{\ensuremath{\left (#1\right)}}
\providecommand{\lbrak}[1]{\ensuremath{\left(#1\right.}}
\providecommand{\rbrak}[1]{\ensuremath{\left.#1\right)}}
\providecommand{\cbrak}[1]{\ensuremath{\left\{#1\right\}}}
\providecommand{\lcbrak}[1]{\ensuremath{\left\{#1\right.}}
\providecommand{\rcbrak}[1]{\ensuremath{\left.#1\right\}}}
\newenvironment{amatrix}[1]{%
  \left(\begin{array}{@{}*{#1}{c}|c@{}}
}{%
  \end{array}\right)
}
\theoremstyle{remark}
\newtheorem{rem}{Remark}
\newtheorem{theorem}{Theorem}[section]
\newtheorem{problem}{Problem}
\newtheorem{proposition}{Proposition}[section]
\newtheorem{lemma}{Lemma}[section]
\newtheorem{corollary}[theorem]{Corollary}
\newtheorem{example}{Example}[section]
\newtheorem{definition}[problem]{Definition}
\newcommand{\sgn}{\mathop{\mathrm{sgn}}}
\providecommand{\abs}[1]{\left\vert#1\right\vert}
\providecommand{\res}[1]{\Res\displaylimits_{#1}} 
\providecommand{\norm}[1]{\left\lVert#1\right\rVert}
%\providecommand{\norm}[1]{\lVert#1\rVert}
\providecommand{\mtx}[1]{\mathbf{#1}}
\providecommand{\mean}[1]{E\left[ #1 \right]}
\providecommand{\fourier}{\overset{\mathcal{F}}{ \rightleftharpoons}}
%\providecommand{\hilbert}{\overset{\mathcal{H}}{ \rightleftharpoons}}
\providecommand{\system}{\overset{\mathcal{H}}{ \longleftrightarrow}}
	%\newcommand{\solution}[2]{\textbf{Solution:}{#1}}
%\newcommand{\solution}{\noindent \textbf{Solution: }}
\newcommand{\cosec}{\,\text{cosec}\,}
\providecommand{\dec}[2]{\ensuremath{\overset{#1}{\underset{#2}{\gtrless}}}}
\newcommand{\myvec}[1]{\ensuremath{\begin{pmatrix}#1\end{pmatrix}}}
\newcommand{\myaugvec}[2]{\ensuremath{\begin{amatrix}{#1}#2\end{amatrix}}}
\newcommand{\mydet}[1]{\ensuremath{\begin{vmatrix}#1\end{vmatrix}}}
\newcommand\figref{Fig.~\ref}
\newcommand\appref{Appendix~\ref}
\newcommand\tabref{Table~\ref}
\newcommand{\romanNumeral}[1]{\uppercase\expandafter{\romannumeral#1}}
%\numberwithin{equation}{section}
%\numberwithin{equation}{subsection}
%\numberwithin{problem}{section}
%\numberwithin{definition}{section}
%\makeatletter
%\@addtoreset{figure}{problem}
%\makeatother

%\let\StandardTheFigure\thefigure
\def\inputGnumericTable{}                                 %%
%New macro definitions
\newcounter{matchleft}\newcounter{matchright}

\newenvironment{matchtabular}{%
  \setcounter{matchleft}{0}%
  \setcounter{matchright}{0}%
  \tabularx{\textwidth}{%
    >{\leavevmode\hbox to 1.5em{\stepcounter{matchleft}\arabic{matchleft}.}}X%
    >{\leavevmode\hbox to 1.5em{\stepcounter{matchright}\alph{matchright})}}X%
    }%
}{\endtabularx}

\title{Matrix assignment}
\date{\today}
\begin{document}
\begin{enumerate}   
	\item   Find the order of the differential equation of the family of circles of radius 3 units.
	\item Find the angle between the line $\overrightarrow{r} = \brak{2\hat{i} - \hat{j} + 3\hat{k}} + \lambda\brak{3\hat{i} - \hat{j} + 2\hat{k}}$ and the plane $\overrightarrow{r} . \brak{\hat{i} + \hat{j} + \hat{k}} = 3$.
	\item Find the co-ordinates of the point, where the line $\frac{x+2}{1} = \frac{y-5}{3} = \frac{z+1}{5}$ cuts the yz-plane.
	\item If $A = \begin{bmatrix}
			8 & 2\\
			3 & 2
	\end{bmatrix}$, then find $\begin{vmatrix}
		adj A
	\end{vmatrix}$.
\item If $y = 2\sqrt{sec\brak{e^{2x}}}$, then find $\frac{dy}{dx}$.
\item Four cards are drawn one by one with replacement from a well-shuffled deck of playing cards. Find the porbability that at least three cards are of diamonds.
\item The probability of two students A and B coming to school on time are $\frac{2}{7}$ and $\frac{4}{7}$, respectively. Assuming that the events `A coming on time' and `B coming on time' are independent, find the probability of only one of them coming to school on time. 
\item Using vectors, prove that the points $\brak{2, -1, 3}, \brak{3, -5, 1} \text{and} \brak{-1, 11, 9}$ are collinear.
\item  For any two vectors $\overrightarrow{a} \text{and} \overrightarrow{b}$, prove that 

	                             ${\brak{\overrightarrow{a} \times \overrightarrow{b}}}^{2} = {\overrightarrow{a}}^{2}  {\overrightarrow{b}}^{2} - {\brak{\overrightarrow{a}\text{ . } \overrightarrow{b}}}^{2}$
	\item Find :

		                   $\int\frac{x - 1}{\brak{x - 2}\brak{x - 3}}dx$
\item  Integrate :


	$\frac{e^x}{5 - 4e^x - e^{2x}}$ with respect to x.
\item If $P\brak{A} = 0.6, P\brak{B} = 0.5 \text{and} P\brak{B|A} = 0.4\text{, find} P\brak{A \bigcup B} \text{and} P\brak{A|B}$.
\item  Find the value of \brak{x - y} from matrix equation


	$2\begin{bmatrix}
		x & 5\\
		7 & y - 3
	\end{bmatrix}
		+ \begin{bmatrix}
		-3 & -4\\
		1 & 2
	\end{bmatrix}
	= \begin{bmatrix}
		7 & 6\\
		15 & 14
	\end{bmatrix}$
\item  Find:


	$\int e^x\brak{\cos x - \sin x}\cosec^2 x dx$
\item Find the differetial equation of the family of curves represented by $y^2 = a\brak{b^2 - x^2}$.
\item  Let an operation $\ast$ on the set of natural numbers N be defined by $a \ast b = a^b$. Find \brak{i} whether $\ast$ is a binary or not, and \brak{ii} if it is a binary, then is it commutative or not. 
\item  Find the particular solution of the differential equation :


	$\brak{1 + e^{2x}}dy + \brak{1 + y^{2}}e^xdx = 0\text{, given that } y\brak{0} = 1$.
\item Find the particular solution of the differential equation :


	$x\frac{dy}{dx}\sin\brak{\frac{y}{x}} + x - y\sin\brak{\frac{y}{x}} = 0\text{, given that } y\brak{1} = \frac{\pi}{2}$.
\item Prove that the relation R in the set A = \cbrak{1, 2, 3, 4, 5, 6, 7} given by $R = \cbrak{\brak{a, b} : \mydet{ a - b} \text{is even}}$ is an equivalence relation.
\item  Show that the funtion f in $A = R - \cbrak{\frac{2}{3}} \text{ defined as }f\brak{x}=\frac{4x + 3}{6x - 4}$ is one-one and onto. Hence, find $f^{-1}$.
\item  Find whether the functin $f\brak{x} = \cos\brak{2x + \frac{\pi}{4}}$; is increasing or decreasing in the interval $\frac{3\pi}{8} < x < \frac{5\pi}{8}$.
\item  Using vectors, find the value of x such that the four points A\brak{x, 5, -1}, B\brak{3, 2, 1}, C\brak{4, 5, 5} and D\brak{4, 2, -2} are coplanar.
\item  Prove that :


	$\sin^{-1}\frac{4}{5} + \tan^{-1}\frac{5}{12} + \cos^{-1}\frac{63}{65} = \frac{\pi}{2}$
\item  If $x^p{y}^{q} = {\brak{x + y}}^{p + q}\text{, prove that }\frac{dx}{dy} = \frac{y}{x} \text{ and } \frac{d^2y}{dx^2} = 0$.
\item  Find : 


	$\int \brak{\sin x . \sin 2x . \sin 3x}dx$
\item  Differentiate $\tan^{-1}\frac{3x - x^3}{1 - 3x^2}, \mydet{x} < \frac{1}{\sqrt{3}} \text{ w.r.t. } \tan^{-1}\frac{x}{\sqrt{1 - x^2}}$.
\item  If $\sqrt{1 - x^2} + \sqrt{1 - y^2} = a\brak{x - y}, \mydet{x}< 1, \mydet{y}< 1\text{, show that  }\frac{dy}{dx} = \sqrt{\frac{1 -y^2}{1 - x^2}}$.
\item  Using properties of determinents, prove the following :


	$\begin{vmatrix}
		a^2 & bc & ac + c^2\\
		a^2 + ab & b^2 & ac\\
		ab & b^2 + bc & c^2
	\end{vmatrix}
	= 4a^2b^2c^2$
\item  Evaluate :


	$\int_{-1}^2 \mydet{x^3 - x} dx$
\item  Find the equation of planes passing through the intersection of planes $\vec{r} . \brak{2\hat{i} + 6\hat{j}} + 12 = 0 \text{ and } \vec{r} . \brak{3\hat{i} - \hat{j} + 4\hat{k}} = 0$ and are at a unit distance from origin.
\item  A company manufactures two types  of novelty souvenirs made of plywood. Souvenirs of type A require 5 minutes each for cutting and 10 minutes each for assembling. Souvenirs of type B require 8 minutes each for cutting and 8 minutes each for assembling. There are 3 hours 20 minutes available for cutting and 4 hours for assembling. The profit for type A souvenirs is \rupee~{100} each and for type B souvenirs, profit is \rupee~{120} each. How many souvenirs of each type should the company manufacture in order to maximise the profit ? Formulate the problem as a LLP and then solve it graphically.
\item  Using integration, find the area of the following region :
\begin{align*}
\cbrak{\brak{x, y} : x^2 + y^2 \leq 16a^2 \text{ and } y^2 \leq 6ax}
\end{align*}
\item  Using integration, find the area of triangle ABC bounded by the lines $4x-y+5=0, x+y-5=0 \text{ and }x-4y+5=0$.
\item  Find the vector equation of the line passing through \brak{2, 1, -1} and parallel to the line $\vec{r}=\brak{\hat{i}+\hat{j}}+\lambda\brak{2\hat{i}-\hat{j}+\hat{k}}$. Also, find the distance between these two line Also, find the distance between these two lin Also, find the distance between these two lines.
\item  Find the coordinates of the foot Q of the perpendicular drawn from the point P\brak{1, 3, 4} to the plane $2x-y+z+3=0$.Find the distance PQ and the image of P treating the plane as a mirror.
\item  Using elementary row transformations, find the inverse of the matrix $\begin{bmatrix}
		3 & 0 & -1\\
		2 & 3 & 0\\
		0 & 4 & 1
\end{bmatrix}$.
\item Using matrices, solve the following system of linear equations : 
\begin{align*}
2x+3y+10z=4\\4x-6y+5z=1\\6x+9y-20z=2
\end{align*}
\item The sum of the perimeters of circle and a square is K, where K is some constant. Prove that the sum of their areas is least when the side of the square is twice the radius of the circle.
\item A bag contains 5 red and 3 black balls and another bag contains 2 red and 6 black balls. Two balls are drawnat random \brak{without replacement} from one of the bags and both are found to be red. Find the probability that balls are drawn from the first bag.  
\end{enumerate}
\end{document}


